%% Generated by Sphinx.
\def\sphinxdocclass{report}
\documentclass[a4paper,10pt,oneside,french]{sphinxmanual}
\ifdefined\pdfpxdimen
   \let\sphinxpxdimen\pdfpxdimen\else\newdimen\sphinxpxdimen
\fi \sphinxpxdimen=.75bp\relax

\usepackage[utf8]{inputenc}
\ifdefined\DeclareUnicodeCharacter
 \ifdefined\DeclareUnicodeCharacterAsOptional
  \DeclareUnicodeCharacter{"00A0}{\nobreakspace}
  \DeclareUnicodeCharacter{"2500}{\sphinxunichar{2500}}
  \DeclareUnicodeCharacter{"2502}{\sphinxunichar{2502}}
  \DeclareUnicodeCharacter{"2514}{\sphinxunichar{2514}}
  \DeclareUnicodeCharacter{"251C}{\sphinxunichar{251C}}
  \DeclareUnicodeCharacter{"2572}{\textbackslash}
 \else
  \DeclareUnicodeCharacter{00A0}{\nobreakspace}
  \DeclareUnicodeCharacter{2500}{\sphinxunichar{2500}}
  \DeclareUnicodeCharacter{2502}{\sphinxunichar{2502}}
  \DeclareUnicodeCharacter{2514}{\sphinxunichar{2514}}
  \DeclareUnicodeCharacter{251C}{\sphinxunichar{251C}}
  \DeclareUnicodeCharacter{2572}{\textbackslash}
 \fi
\fi
\usepackage{cmap}
\usepackage[T1]{fontenc}
\usepackage{amsmath,amssymb,amstext}
\usepackage{babel}
\usepackage{times}
\usepackage[Sonny]{fncychap}
\usepackage[dontkeepoldnames]{sphinx}

\usepackage{geometry}

% Include hyperref last.
\usepackage{hyperref}
% Fix anchor placement for figures with captions.
\usepackage{hypcap}% it must be loaded after hyperref.
% Set up styles of URL: it should be placed after hyperref.
\urlstyle{same}

\addto\captionsfrench{\renewcommand{\figurename}{Fig.}}
\addto\captionsfrench{\renewcommand{\tablename}{Tableau}}
\addto\captionsfrench{\renewcommand{\literalblockname}{Code source}}

\addto\captionsfrench{\renewcommand{\literalblockcontinuedname}{continued from previous page}}
\addto\captionsfrench{\renewcommand{\literalblockcontinuesname}{continues on next page}}

\addto\extrasfrench{\def\pageautorefname{page}}

\setcounter{tocdepth}{3}
\setcounter{secnumdepth}{3}


\title{Diacamma Syndic}
\date{janv. 29, 2019}
\release{2.3.6}
\author{sd-libre}
\newcommand{\sphinxlogo}{\sphinxincludegraphics{DiacammaSyndic.jpg}\par}
\renewcommand{\releasename}{Version}
\makeindex

\begin{document}

\maketitle
\sphinxtableofcontents
\phantomsection\label{\detokenize{index::doc}}



\chapter{Diacamma Syndic}
\label{\detokenize{syndic/index::doc}}\label{\detokenize{syndic/index:diacamma-syndic}}\label{\detokenize{syndic/index:sommaire}}
Présentation du logiciel Diacamma Syndic.


\section{Présentation}
\label{\detokenize{syndic/presentation::doc}}\label{\detokenize{syndic/presentation:presentation}}

\subsection{Description}
\label{\detokenize{syndic/presentation:description}}
\sphinxstyleemphasis{Diacamma Syndic} est un logiciel de gestion spécialement conçu pour les copropriétés et les syndic bénévoles.
Avec \sphinxstyleemphasis{Diacamma}, donnez à votre structure le logiciel qu’elle mérite! Pas besoin d’être informaticien pour avoir les outils adaptés à votre cas.

L’application de base est entièrement gratuite et vous permet de gérer les accès à vos données et au carnet d’adresses nécessaires à votre copropriété.
Les modules complémentaires vous permettront d’adapter en quelques clics le logiciel à vos besoins.

Les différents modules disponibles vous permettront, par exemple, de:
\begin{itemize}
\item {} 
Identifier les copropriétaires et leurs différents tantièmes, réaliser des appels de fonds correspondant à votre budget, ventiler les factures et frais de la copropriété suivant les parts de chacun.

\item {} 
Gérer vos documents de façon centralisée grâce à la gestion documentaire.

\item {} 
Gérer une comptabilité simple et performante et clôturer rapidement votre exercice financier pour votre assemblée générale.

\end{itemize}

Ce manuel vous aidera dans l’utilisation de ce logiciel.
Si malgré tout, vous ne trouvez pas la réponse à vos problèmes, visiter notre site \sphinxurl{http://www.diacamma.org} où vous trouverez des tutoriels et des astuces.


\subsection{Installation}
\label{\detokenize{syndic/presentation:installation}}
Vous pouvez installer \sphinxstyleemphasis{Diacamma Syndic} sur un ordinateur dédié à votre association que ce soit un Apple Macintosh (OS X 10.8 et +) ou bien un PC sous MS-Windows (7 et +) ou sous GNU Linux (Ubuntu 14.04 ou +).

\sphinxstyleemphasis{Diacamma Syndic} est un logiciel client/serveur : vous pouvez l’installer sur un ordinateur centralisateur et accéder aux données depuis un autre PC connecté au premier, sans limite du nombre d’utilisateurs simultanés.
Si le PC contenant les données est connecté de manière permanente à internet, vous aurez accès à vos données depuis n’importe où dans le monde!

Cette organisation est particulièrement intéressante pour permette à plusieurs cadres associatifs d’avoir accès à des données communes.

Quel responsable ne s’est pas arraché les cheveux suite à un échanges de documents via une clef USB où certaines modifications importantes se perdent?

Pour plus d’information, visiter notre site \sphinxurl{http://www.diacamma.org}


\section{Prise en main}
\label{\detokenize{syndic/first_step::doc}}\label{\detokenize{syndic/first_step:prise-en-main}}
Le logiciel \sphinxstyleemphasis{Diacamma Syndic} comprends un grand nombre de paramétrages et peu paraître difficile à configurer au besoin de votre structure.

Nous vous proposons cette explication pour vous aider à franchir cette première étape dans l’utilisation de cet outil.

Suivez pas à pas les différents phases de réglages. Dans chaque étape, nous ne ré-détaillons pas les fonctionnalités. Nous vous invitons également à vous référer au reste du manuel utilisateur pour cela.

Il peut être intéressant de réaliser des sauvegardes au cours de cette procédure.
Cela vous permettra, si vous faite une erreur, de revenir à une étape précédente sans tout recommencer (installation comprise).


\subsection{Vérifiez la mise à jours de votre logiciel}
\label{\detokenize{syndic/first_step:verifiez-la-mise-a-jours-de-votre-logiciel}}
Commencez par vérifiez que votre logiciel est à jours.
En effet, nous diffusons régulièrement des correctifs qui ne sont pas toujours inclus dans les installateurs.


\subsection{Présentation de votre copropriété}
\label{\detokenize{syndic/first_step:presentation-de-votre-copropriete}}\begin{quote}

Menu \sphinxstyleemphasis{General/Nos coordonnées}
\end{quote}

Dans cette écran, vous pouvez décrire les coordonnées de votre structure.
De nombreuses fonctionnalités utilisent ces informations en particulier pour les impressions.


\subsection{Création de votre premier exercice comptable}
\label{\detokenize{syndic/first_step:creation-de-votre-premier-exercice-comptable}}\begin{quote}

Menu \sphinxstyleemphasis{Administration/Modules (conf.)/Configuration comptables}

Menu \sphinxstyleemphasis{Finance/Comptabilité/plan comptable}
\end{quote}

Ouvrer votre premier exercice comptable et rendez le actifs.
Vous devrez aussi créer le plan comptable de cette exercice pour avoir une comptabilité pleinement opérationnel.


\subsection{Réglage de votre copropriété}
\label{\detokenize{syndic/first_step:reglage-de-votre-copropriete}}\begin{quote}

Menu \sphinxstyleemphasis{Administration/Modules (conf.)/Configuration de copropriété}
\end{quote}

Vérifiez que ces paramètres correspondent à votre utilisation.


\subsection{Définition des catégories de charges, des lots et des tantièmes}
\label{\detokenize{syndic/first_step:definition-des-categories-de-charges-des-lots-et-des-tantiemes}}\begin{quote}

Menu \sphinxstyleemphasis{Copropriété/Les propriétaires et les lots}

Menu \sphinxstyleemphasis{Copropriété/Les catégories de charges}
\end{quote}

Déclarez vos copropriétaires ainsi que les lots de votre copropriétés.
Ajoutez vos catégories de charges et définisez pour chacunes les cotes parts ou les lots associés.


\subsection{Mise à jours comptable}
\label{\detokenize{syndic/first_step:mise-a-jours-comptable}}\begin{quote}

Menu \sphinxstyleemphasis{Finance/Comptabilité/écritures comptable}

Menu \sphinxstyleemphasis{Finance/Comptabilité/Modèle d’écriture}
\end{quote}

Si vous mettez en place \sphinxstyleemphasis{Diacamma Syndic} au cours de votre exercice, vous devrez également saisir votre report à nouveau de l’exercice précédent ainsi que ressaisir les écritures du début d’année.
Attention: n’oubliez pas que l’ajout d’une cotisation d’adhérent génère une facture ainsi que des écritures comptables associées. Prenez en compte dans la reprise de votre comptabilité.
Pour vous aidez dans la saisie de votre comptabilité, nous vous conseillons d’utiliser les modèles d’écritures. Enregistrez entant que modèle les écritures récurrentes que vous avez au cours d’une année. Ainsi vous pouvez rapidement compléter votre comptabilité en quelques cliques.


\subsection{Le courriel}
\label{\detokenize{syndic/first_step:le-courriel}}\begin{quote}

Menu \sphinxstyleemphasis{Administration/Modules (conf.)/Paramètrages de courriel}
\end{quote}

Définissez vos réglages pour votre courriel.
Le serveur smpt permettra à \sphinxstyleemphasis{Diacamma Syndic} d’envoyé un certain nombre de message: facture en PDF, mot de passe de connexion, …
Vous pouvez préciser comment réagis les liens “écrire à tous” réagis avec votre logiciel de messagerie.


\subsection{Les responsables}
\label{\detokenize{syndic/first_step:les-responsables}}\begin{quote}

Menu \sphinxstyleemphasis{Général/Nos coordonnées}

Menu \sphinxstyleemphasis{Administration/Modules (conf.)/Configuration des contacts}

Menu \sphinxstyleemphasis{Association/Adhérents/Adhérents cotisants}
\end{quote}

Dans la fenêtre de vos coordonnées, vous pouvez associé des personnes comme résponsable de votre structure comme les copropriétaire membre du conseil d’administration.
Utilisez l’outil de recherche et assignez leur une fonction.
Vous pouvez également rajouter des fonctions propres à votre structure.

Depuis la fiche de chacun de vos copropriétaire, vous pouvez donner à des personnes actives un droit de connexion à Diacamma Syndic.
Privilégié une utilisation du logiciel avec un alias et un mot de passe propre à chaque utilisateur. Associez leur également des droits correspondants à leurs fonctions au sein de votre structure.
Enfin, évitez autant que possible l’utilisation de l’alias “admin” qui doit être réservé pour des actions de configuration et de maintenance.


\subsection{La gestions documentaire}
\label{\detokenize{syndic/first_step:la-gestions-documentaire}}\begin{quote}

Menu \sphinxstyleemphasis{Administration/Modules (conf.)/Dossier}

Menu \sphinxstyleemphasis{Bureautique/Documents/Documents}
\end{quote}

Définissez vos différents dossier vous permettant d’importer vos documents à classer et à partager.

Une fois avoir parcouru ces points, votre logiciel \sphinxstyleemphasis{Diacamma Syndic} est pleinement opérationnel.
N’hésistez pas à consulter le forum: de nombreuses astuces peux vous aider pour utiliser au mieux votre logiciel.


\chapter{Diacamma copropriété}
\label{\detokenize{condominium/index::doc}}\label{\detokenize{condominium/index:diacamma-copropriete}}
Aide relative aux fonctionnalités de gestion de copropriété.


\section{Les copropriétaires}
\label{\detokenize{condominium/owners:les-coproprietaires}}\label{\detokenize{condominium/owners::doc}}
Déclarer vos copropriétaires.
Comptablement, un copropriétaire est égalemet un tiers comptable possédant des comptes sociétaires (classe “45xx” dans le plan français).

Vous pouvez également voir ici un résumé de la situation financière de chacun des copropriétaires.
La fiche de chaque copropriétaire vous présente une situtation financière précise.


\subsection{Les lots}
\label{\detokenize{condominium/owners:les-lots}}
Déclarer les lots de votre copropriétés.
Précisez en plus de la description du lot, les tantièmes et le propriétaires de chacun.
\begin{quote}

\noindent\sphinxincludegraphics{{set_owner}.png}
\end{quote}


\section{Les catégories de charges}
\label{\detokenize{condominium/classloads:les-categories-de-charges}}\label{\detokenize{condominium/classloads::doc}}
Depuis le menu \sphinxstyleemphasis{Copropriété/Gestion/Les catégories de charges}, vous pouvez définir des ensembles de lots où se répartisse différement des dépenses suivant chaque copropriétaires.
Par exemple, vous pouvez distinguer en catégories de charges: les montés d’excalier, les garages, le chauffage, l’eau, …


\subsection{Charges courantes et charges exceptionnels}
\label{\detokenize{condominium/classloads:charges-courantes-et-charges-exceptionnels}}
Une catégorie de charge doit être défini entant que courante ou exceptionnels.
La première catégorie défini des frais récurent budgetisable chaque année.
La deuxième correspond à une opération ponctuelle (comme des travaux) qui seront cloturées une fois l’opération terminée.
La réglementation française demande de séparer la comptabilité entre ces deux différentes catégories.


\subsection{La répartition}
\label{\detokenize{condominium/classloads:la-repartition}}
Deux modes de répartition peuvent être utilisé pour chaque catégorie de charges.
* Liée au lots
Dans ce cas, assignez un ensemble de lot sur cette catégorie: les tantièmes correspondants seront automatiquement associés au différents propriétaires afin de définir leurs répartitions.
* Non liée au lots
Vous pouvez, dans ce cas, définir au cas par cas, pour chaque propriétaire les tantièmes et la répartition pour cette catégorie.
Ce mode de répartition est utilisé dans le cas de gestion sur des consomations fixes (surface, étage, …) ou variables (consomation).


\subsection{Les budgets prévisionnels}
\label{\detokenize{condominium/classloads:les-budgets-previsionnels}}
Depuis l’interface d’une catégorie de charges, vous pouvez ajouter un budget prévisionnel.
Cliquez simplement sur le bouton \sphinxstyleemphasis{Budget} depuis sa fiche.

L’interface vous permet alors d’ajouter des comptes de charges ou de produits ainsi qu’un solde prévisionnel.
vous pouvez également importer les montants des charges et produits du résultat d’une année précédente.

Ce budget prévisionnel apparait alors dans les rapports afin de le comparer avec la comptabilité réalisé.


\section{Les appels de fonds}
\label{\detokenize{condominium/call_of_funds:les-appels-de-fonds}}\label{\detokenize{condominium/call_of_funds::doc}}
Pour créer un appel de fonds, cliquez sur le menu \sphinxstyleemphasis{copropriete/Gestion/Les appels de fonds}.
Depuis cette liste des appels de fond, vous pouvez en créer un nouveau via le bouton « Ajouter ».

Une fois avoir précisé la date et le descriptif de cette appel, une nouvelle fiche d’appel est créé.
Ajoutez dans cette fiche différentes élements d’appel.
Sur chaqu’un vous préciser l’ensemble de lots associé à cet élement d’appel ainsi que le montant.
\begin{quote}

\noindent\sphinxincludegraphics{{call_of_funds}.png}
\end{quote}

Enfin, pour finaliser l’appel de fond, cliquez sur « Valider »
L’ensemble des copropriétaires se voient alors associés à une nouvelle fiche d’appel de fond.
Les montants de chaques éléments précédement saisi sont modifiés en fonction du ratio de l’ensemble que possède chaque copropriétaire.


\section{Le suivi des copropriétaires}
\label{\detokenize{condominium/payoff::doc}}\label{\detokenize{condominium/payoff:le-suivi-des-coproprietaires}}
Suite à un appel de fond, les copropriétaires transmettent leurs payements au syndic.
Pour le saisir, allez dans la fiche de tiers du copropriétaire depuis la liste des copropriétaire de \sphinxstyleemphasis{copropriété/Gestion/Les proprietaires et les lots}.
\begin{quote}

\noindent\sphinxincludegraphics{{payoff}.png}
\end{quote}

Ajoutez alors un payement en précisant la date, le montant, un référence de payement (comme le numéro de chèque) ainsi que le compte comptable à imputer.
Les écritures en comptabilité sont alors automatiquement réalisé en brouillard.
La situation du copropriétaire est également mis à jours.


\section{Les dépenses}
\label{\detokenize{condominium/expense::doc}}\label{\detokenize{condominium/expense:les-depenses}}
Pour gérer une dépense de copropriété, cliquez sur le menu \sphinxstyleemphasis{copropriete/Gestion/Les dépenses}.
Depuis cette liste des dépenses, vous pouvez en créer un nouveau via le bouton « Ajouter ».

Une fois avoir précisé la date et le descriptif de cette dépense, une nouvelle fiche d’appel est créé.
Commencez par indiquer le fournisseur de votre nouvelle dépense. Celui-ci doit être un tiers référencé.
Comme pour les appel de fonds, ajoutez dans cette fiche différentes élements de dépense. Sur chaqu’un vous préciser l’ensemble de lots associé ainsi que son montant.
\begin{quote}

\noindent\sphinxincludegraphics{{expense}.png}
\end{quote}

Enfin, pour finaliser la dépense, cliquez sur « Valider »
Un certain nombre d’action sont alors réalisé en comptabilité:
\begin{itemize}
\item {} 
L’écriture de dépense est généré en brouillard.

\item {} 
Pour chaque copropriétaire, une ventillation de cette dépense est réalisé en fonction de leurs ratio de l’ensemble que possède chaque copropriétaire.

\end{itemize}

Depuis cette fiche de dépense, vous pouvez aussi entrer les réglements de votre fournisseur.
Une écriture correspondante sera également généré.


\section{Les rapports}
\label{\detokenize{condominium/report::doc}}\label{\detokenize{condominium/report:les-rapports}}
Vous retrouvez ici les rapports correspondant au 4 premières annexes de la réglementation française.
L’annexe 5 n’est pas proposé actuellement dans Diacamma: correspondant à un suivi de travaux et pas seulement un rapport financier, il nécessite une gestion de taches afin d’établir un tel rapport.


\subsection{État financier}
\label{\detokenize{condominium/report:etat-financier}}
Ce rapport présente la situation financière de la copropriété:
* La trésorerie
* Les fonds d’attente
* Les dettes fournisseurs
* La situation des copropriétés


\subsection{Compte de gestion générale}
\label{\detokenize{condominium/report:compte-de-gestion-generale}}
Présente, de façon général, l’ensemble des charges et produits courantes de l’exercices.


\subsection{Compte de gestion pour opérations courantes}
\label{\detokenize{condominium/report:compte-de-gestion-pour-operations-courantes}}
Présente, par catégories de charges courantes, le solde des différents charges engagés.


\subsection{Compte de gestion pour opérations exceptionnelles}
\label{\detokenize{condominium/report:compte-de-gestion-pour-operations-exceptionnelles}}
Présente, par catégories de charges exceptionnelles, le solde des différents charges engagés.


\chapter{Diacamma comptabilité}
\label{\detokenize{accounting/index:diacamma-comptabilite}}\label{\detokenize{accounting/index::doc}}
Aide relative aux fonctionnalités comptables.


\section{Definitions}
\label{\detokenize{accounting/definition::doc}}\label{\detokenize{accounting/definition:definitions}}\begin{quote}

\sphinxstylestrong{Remarques:} Ce module comptable est proche d’une comptabilité type « entreprise », néanmoins elle ne respecte pas certaines exigences légales et fiscale en la matiére.
Ce modules ne peux pas étre utilisé pour la tenu de compte de structures commerciales, concurrentielles ou professionnelles mais seulement des structures de type associative gérées par des bénévoles.
Le représentant légale de la structure utilisant ce module doit s’assurer que sa comptabilité respecte alors la législation de son pays en vigueur.
\end{quote}


\subsection{Exercice comptable}
\label{\detokenize{accounting/definition:exercice-comptable}}
Un exercice comptable est une période de temps sur laquelle une
personne morale (entreprise, association ou autre) enregistre tous les
mouvements d’argent la concernant.

Cette période est généralement de 12 mois consécutifs du 1er janvier au 31 décembre mais peut varier
d’une entité à une autre. La durée légale est toutefois fixée à un
maximum de 2 ans. La durée de l’exercice est fixée à l’avance et ne
peut être modifiée que sur décision du conseil d’administration.


\subsection{Tiers comptable}
\label{\detokenize{accounting/definition:tiers-comptable}}
Un tiers comptable est une personne physique ou morale avec
laquelle une entité va avoir des échanges monétaires (clients,
fournisseurs, salariés, administrations…).


\subsection{Journaux comptables}
\label{\detokenize{accounting/definition:journaux-comptables}}
Un journal comptable est un regroupement d’écritures comptables permettant de classer celles-ci.

Les journaux par défaut sont:
* journal d’achat contenant toutes les écritures relatives aux achats fait par une entité
* journal de vente contenant toutes les écritures relatives aux dépenses effectuées par une entité
* journal des encaissements contenant toutes les écritures relatives aux mouvement sur les comptes en monétaire (compte bancaires, compte caisse…) en relation avec les dépenses et recettes de l’entité
* journal des reports à nouveau contenant les écritures permettant le passage d’un exercice comptable à son suivant
* journal des opérations diverses contenant l’ensemble des autres écritures (ex: frais financiers…)


\subsection{Ecritures comptables}
\label{\detokenize{accounting/definition:ecritures-comptables}}
Une écriture comptable est un ensemble de lignes inscrites dans divers
comptes comptables permettant un équilibre.
La somme des crédits d’une écriture doit donc être égale à la somme des débits de cette même écriture.

Par exemple, une écriture d’achat se schématise par:
* une ligne au crédit du compte tiers fournisseur représentant l’ensemble de la somme de la facture
* une ou plusieurs lignes au débit des comptes de charges correspondants au type de ressources achetées (matériel, service…)

Le total des lignes dans les comptes de charge est donc égal au montant
porté sur la ligne de compte tiers fournisseur.

Les écritures comptables d’encaissement peuvent et doivent être pointées afin de marquer le rapprochement avec les comptes bancaires et
la caisse physique. De cette façon, on peut suivre facilement les écritures passées dans la comptabilité mais non encore effectives dans
la réalité. Le pointage est nécessaire pour le passage d’un exercice comptable à son suivant.

Il est également possible et recommandé de lettrer les écritures, c’est à dire de créer un sous ensemble
cohérent d’écritures en provenance de journaux divers afin de stipuler qu’elle correspondent à la même opération de la vie réelle.

Ex: l’écriture comptable d’achat d’un bien peut être lettrée avec son écriture d’encaissement.


\subsection{Plan comptable de l’exercice}
\label{\detokenize{accounting/definition:plan-comptable-de-l-exercice}}
Le plan comptable de l’exercice est l’ensemble des comptes utilisés au
cours d’un exercice comptable en se basant sur le plan comptable
couramment admis par l’administration fiscale.

les numéros de comptes doivent impérativement commencer par le préfixe donné par le
plan comptable en vigueur au moment de la création du compte.


\section{Exercice}
\label{\detokenize{accounting/fiscalyear::doc}}\label{\detokenize{accounting/fiscalyear:exercice}}

\subsection{Paramétrages}
\label{\detokenize{accounting/fiscalyear:parametrages}}\begin{quote}

\noindent\sphinxincludegraphics{{parameters}.png}
\end{quote}

Initialement, vous pouvez définir ici le type de système comptable que
vous voulez utiliser (ex: Plan comptable générale Français).
\sphinxstyleemphasis{Attention:} une fois défini, ce système n’est plus modifiable.

Vous pouvez changer également la monnaie courrante de votre comptabilité.


\subsection{Création d’un exercice comptable}
\label{\detokenize{accounting/fiscalyear:creation-d-un-exercice-comptable}}
Pour créer un exercice comptable, rendez vous dans le menu \sphinxstyleemphasis{Administration/Modules (conf.)/Configuration comptable}.
\begin{quote}

\noindent\sphinxincludegraphics{{fiscalyear_list}.png}
\end{quote}

De là, cliquez sur Ajouter afin de faire apparaître le formulaire vous permettant de renseigner les bornes de l’exercice
\begin{quote}

\noindent\sphinxincludegraphics{{fiscalyear_create}.png}
\end{quote}

Indiquez la date de début (celle-ci doit étre le lendemain de la date
de clôture de l’exercice précédent) et la date de fin (au maximum 2 ans
aprés le début de l’exercice) de l’exercice puis cliquez sur le bouton
OK.

Votre nouvel exercice sera alors disponible dans la
liste des exercices. Pour en continuer la création, il vous faudra le
sélectionner dans la liste et cliquer sur le bouton Activer afin de
pouvoir travailler dessus par défaut.

Depuis ce même écran de configuration, vous pouvez également modifier
ou ajouter des journaux.
Vous pouvez également créer des champs personnalisés (comme pour la fiche de contacte)
pour la fiche de tiers. Ceci peut être interessant si vous voulez réaliser des recherches/filtrages
sur des informations propres à votre fonctionnement.

Fermez maintenant la liste des exercices afin de vous rendre dans la comptabilité et
pouvoir créer le plan comptable de l’exercice ainsi qu’affecter le
report à nouveau avant de pouvoir commencer éà saisir des écritures.

Pour ce faire, rendrez-vous dans le menu \sphinxstyleemphasis{Financier/Comptabilité/plan   comptable}
\begin{quote}

\noindent\sphinxincludegraphics{{account_list}.png}
\end{quote}

Ici, commencez par créer les comptes de base de votre exercice.

Si vous avez déjà un précédent exercice, vous pouvez en importer la liste de code comptable.
\begin{description}
\item[{Une fois ceci fait, plusieurs choix se présentent à vous:}] \leavevmode\begin{itemize}
\item {} \begin{description}
\item[{Il s’agit de votre premier exercice comptable}] \leavevmode
Vous venez de créer votre structure, vous n’avez pas de report à nouveau, cliquez donc dès maintenant sur le bouton commencer, vous aurez alors achevé la création de votre exercice.

\end{description}

\item {} \begin{description}
\item[{Il ne s’agit pas de votre premier exercice mais vous n’utilisiez pas ce logiciel avant}] \leavevmode
Il va falloir saisir manuellement votre report à nouveau.
Pour cela, sortez du plan comptable de l’exercice, entrez dans la liste des écritures et saisissez manuellement une écriture complète, cohérente et équilibrée pour votre report à nouveau.
Une fois ceci fait, retournez dans votre plan comptable pour cliquer sur commencer.

\end{description}

\item {} \begin{description}
\item[{Il ne s’agit pas de votre premier exercice et vous utilisiez déjà ce logiciel}] \leavevmode
Utilisez le bouton « report à nouveau » » afin d’importer le résultat de l’exercice précédent.
Comme il n’est pas possible de commencer un exercice avec un résultat (qu’il soit bénéficiaire ou déficitaire).
Vous devez avant de commencer votre exercice, ventiler cette somme sur un compte de capitaux (capital, réserve, ..).
La décision de cette affectation est prise par le conseil d’administration sous le contrôle de votre vérificateur aux comptes.
Pour cela, vous pouvez créer une écriture spécifique (journal “report à nouveau”) ou utilisez le questionnaire l’or du commencement de l’exercice.
Pour commencer l’exercice, clique sur le bouton afin de clore cette phase de création.

\end{description}

\end{itemize}

\end{description}


\subsection{Création, modification et édition de comptes dans le plan}
\label{\detokenize{accounting/fiscalyear:creation-modification-et-edition-de-comptes-dans-le-plan}}
Plaçons nous dans le menu \sphinxstyleemphasis{finance/comptabilité/plan comptable de l’exercice}.

A tout moment au cours d’un exercice vous pouvez être amener à ajouter un nouveau compte dans votre plan.
\begin{quote}

\noindent\sphinxincludegraphics{{account_new}.png}
\end{quote}

Référez vous aux codes légaux définis par la réglementation de votre pays pour définir correctement les 3 premiers chiffres.
Pour les associations dépendant du droit français, vous pourrez trouver des informations sur le site du gouvernement des finances français (\sphinxurl{http://www.minefe.gouv.fr/themes/entreprises/compta\_entreprises/index.htm}).
Les 3 derniers chiffres du compte vous sont propres suivant votre besoin. Modifiez la désignation pour simplifier l’identification de votre compte.

Si vous vous étes trompé, vous pouvez changer le compte et sa désignation. Si des écritures ont été saisies avec ce compte, elles seront automatiquement migrées.

Par contre, le nouveau compte doit rester dans la même catégorie comptable.
Vous pouvez consulter un compte précis. Vous pouvez alors voir
l’ensemble des lignes d’écritures associées à ce compte, ainsi que la
valeur du compte au début (report à nouveau) et la valeur actuelle.
\begin{quote}

\noindent\sphinxincludegraphics{{account_edit}.png}
\end{quote}

Il vous est aussi possible de supprimer un compte du plan si aucune opération n’y a été réalisée.


\subsection{Clôture d’un exercice}
\label{\detokenize{accounting/fiscalyear:cloture-d-un-exercice}}
A la fin de la période, vous devez clôturer votre exercice. Cette
opération, définitive, se réalise sous le contrôle de votre
vérificateur aux comptes.
Dans le menu \sphinxstyleemphasis{Financier/Comptabilité/plan comptable}, cliquez sur le bouton « Clôturer ».

\sphinxstylestrong{Attention:} Toutes les écritures doivent être validées avant de commencer cette procédure.

La phase de validation va réaliser un traitement consistant à
créer une série d’écritures de fin d’exercice résumant le résultat et
les dettes tiers (factures clients ou fournisseurs transmises mais pas encore réglées).


\section{Tiers comptable}
\label{\detokenize{accounting/third::doc}}\label{\detokenize{accounting/third:tiers-comptable}}

\subsection{Création d’un tiers}
\label{\detokenize{accounting/third:creation-d-un-tiers}}
Plaçons nous dans le menu \sphinxstyleemphasis{Finance/Tiers}.

\noindent\sphinxincludegraphics{{third_list}.png}

La liste des tiers précédemment enregistrés apparaît.
Vous pouvez réaliser un certain nombre de fitrage rapide suivant le nom ou
la situation. Vous pouvez alors imprimer la liste.
Pour ajouter un nouveau tiers, vous devez commencer par choisir un contact (physique
ou moral) associé à ce tiers comptable.

\noindent\sphinxincludegraphics{{third_add}.png}

Depuis cet écran, vous pouvez aussi créer un nouveau contact avant de le sélectionner.

\noindent\sphinxincludegraphics{{third_edit}.png}

Pour chaque tiers, vous pouvez associer des comptes comptables
correspondant à la nature de vos tiers: fournisseur, client, salarié et
sociétaire. Vous pouvez changer ces comptes pour imputer dans votre
comptabilité comme vous le souhaitez cette personne au cours
d’opérations financières.


\subsection{Situation d’un tiers}
\label{\detokenize{accounting/third:situation-d-un-tiers}}
La fiche d’un tiers vous permet d’avoir une vue globale de l’état des recettes et dépenses liées à ce tiers.

\noindent\sphinxincludegraphics{{third_state}.png}

Vous retrouverez ici l’ensemble des écritures comptables de
l’exercice liées à ce tiers. Vous trouverez également un résumé des
débits et crédits permettant en un seul regard de savoir s’il reste des
dettes impayées. Avec d’autres modules financiers, vous pourrez
également consulter des opérations liées.


\subsection{Configuration}
\label{\detokenize{accounting/third:configuration}}
Depuis le menu \sphinxstyleemphasis{Administration/Modules (conf.)/Configuration comptable} vous avez la possibilité d’ajouter à tout tiers des champs personnalisés.
Le mécanisme est similaire à ce que vous pouvez trouver dans la configuration des contacts.


\section{Ecritures}
\label{\detokenize{accounting/entity:ecritures}}\label{\detokenize{accounting/entity::doc}}

\subsection{Saisie d’une écriture}
\label{\detokenize{accounting/entity:saisie-d-une-ecriture}}\begin{quote}

\sphinxstylestrong{Cas général}
\end{quote}

Plaçons nous dans le menu \sphinxstyleemphasis{Financier/Comptabilité/écritures comptables}.
\begin{quote}

\noindent\sphinxincludegraphics{{entity_list}.png}
\end{quote}

Depuis cet écran, nous avons la possibilité de visualiser les écritures
précédemment saisies ainsi que d’en ajouter de nouvelles.

Comme vous pouvez le voir dans cet écran, vous pouvez consulter les écritures
par journaux ou par état. 5 filtres d’état vous sont proposés:
\begin{itemize}
\item {} 
Tout: aucun filtrage n’est appliqué

\item {} 
En cours (Brouillard): seulement les écritures non encore validées

\item {} 
Validé: seulement les écritures déjé validées

\item {} 
Lettré: seulement les écritures rapprochées ou lettrées avecd’autres

\item {} 
Non lettré: seulement les écritures non encore lettrées

\end{itemize}

Ainsi que 5 journaux par défaut:
* Journal des achats
* Journal des ventes
* Journal des encaissements
* Journal des opérations diverses
* Journal de report à nouveaux

Pour ajouter une écriture, commençons d’abord par sélectionner sur quel
journal nous souhaitons réaliser notre nouvelle écriture, puis cliquons
sur le bouton \sphinxstyleemphasis{Ajouter}.

Aprés avoir précisé les dates de votre écriture, il vous faut
ajouter les différentes lignes correspondant é votre opération financiére.

Pour ajouter une ligne d’écriture, saisissez son code comptable
si vous le connaissez dans la zone d’ajout ou laissez-vous guider par
l’assistant en cliquant sur le bouton correspondant au type de compte désiré.
\begin{quote}

\noindent\sphinxincludegraphics{{entity_add}.png}
\end{quote}

L’outil ne vous permettra pas de valider votre écriture si elle est déséquilibrée.

Quand on débute en comptabilité, on a parfois du mal pour savoir si une ligne est en débit ou en crédit. Pour vous aider, un message
vous avertit si vous avez saisi un remboursement ou un avoir et un bouton vous permettez d’inverser trés facilement votre écriture si besoin.
\begin{quote}

\sphinxstylestrong{Réaliser un encaissement}
\end{quote}

Une écriture d’encaissement peut se saisir manuellement comme précédemment mais bien souvent, un réglement vient compléter un achat ou une vente effectué quelques jours plus tét.

Pour simplifier votre saisie, ré-ouvrez l’écriture d’achat ou de vente dont vous souhaitez saisir le réglement, cliquez sur le bouton « Encaissement »: l’application vous propose alors une nouvelle écriture
partiellement remplie. Il ne vous reste plus qu’é préciser sur quel compte financier (caisse, banque…) vous voulez réaliser cette opération.

Une fois un encaissement validé via ce mécanisme, les deux écritures (celle d’achat ou de vente et celle d’encaissement) sont automatiquement lettrées.
\begin{quote}

\sphinxstylestrong{Ecriture de report à nouveau}
\end{quote}

Le journal de report à nouveau n’est modifiable que dans la phase d’initialisation de votre exercice.

A ce moment, vous pouvez étre amené à réaliser des opérations spécifiques comme par exemple la ventilation des bénéfices de l’année
précédente suivant plusieurs comptes.

Par contre, dans ce journal, il n’est pas possible d’ajouter des lignes d’écritures de charges ou de produits.


\subsection{Lettrage d’écritures}
\label{\detokenize{accounting/entity:lettrage-d-ecritures}}
Comme nous l’avons évoqué dans un précédent chapitre, il est régulier
qu’un ensemble d’écritures se référent à une ou plusieurs opérations
communes. Dans ce cas, vous pouvez marquer ces écritures comme étant
liées: vous allez alors les lettrer.

Le plus souvent, le lettrage
se réalise entre écritures de valeur de tiers complémentaires: entre
une écriture d’achat (ou de vente) et son encaissement associé.

Mais, il peut arriver que nous souhaitions lettrer plus de deux
écritures. Par exemple, vous pouvez vouloir régler 3 factures d’un
fournisseur en une seule fois. Dans ce cas, comme vous ne faites qu’un
seul chéque d’un montant égal é la somme des factures, vous n’aurez
qu’une écriture d’encaissement que vous allez lettrer avec les 3
écritures d’achats. A la relecture de votre comptabilité, il deviendra
alors simple de comprendre qu’il s’agissait d’un réglement multiple.

Pour réaliser cette action, sélectionnez les écritures que vous désirez
lier et cliquez sur le bouton « Lettrer »: Si l’outil les considére comme
étant cohérentes, il réalisera le lettrage symbolisé par un numéro
commun é ces écritures en derniére colonne du journal.

Si vous cliquez à nouveau sur ce bouton, vous avez la possibilité de supprimer
le lettrage de cette écriture ainsi que celui des écritures associées.


\subsection{Validation d’écritures}
\label{\detokenize{accounting/entity:validation-d-ecritures}}
Par défaut, une écriture est saisie au brouillard, c’est é dire dans un
état où elle reste modifiable ou supprimable.

Par contre, il est nécessaire, pour finaliser votre comptabilité, de valider cette
écriture pour entériner votre saisie.

Pour réaliser cette action, sélectionnez les écritures contrôlées et
cliquez sur le bouton « Valider »: L’application affectera alors un
numéro à vos écritures ainsi que la date de validation.

Une fois validée, une écriture devient non modifiable: ce mécanisme assure le
caractére intangible et irréversible de votre comptabilité.
Comme l’erreur est humaine, au lieu de supprimer un écriture valider, il vous faudra
créer une écriture inverse de pour l’annuler.

Cela est utile pour un responsable comptable pour préciser que cette
écriture est vérifiée par rapport au justificatif associé.
Cela sert aussi, dans le cas des écritures d’encaissements, de contrôler que
cette recette ou dépense figure bien sur un relevé de banque.

Pour clôturer un exercice, l’ensemble des écritures doivent étre validées.


\subsection{Recherche d’écriture}
\label{\detokenize{accounting/entity:recherche-d-ecriture}}
Depuis la liste des écritures, le bouton « Recherche » vous permet
de définir des critères de recherche d’écritures comptables.
\begin{quote}

\noindent\sphinxincludegraphics{{entity_search}.png}
\end{quote}

En cliquant sur “Rechercher », l’outil va rechercher dans la base
toutes les écritures correspondantes à ces critères. Vous pourrez alors
imprimer cette liste ou éditer/modifier une écriture.


\section{Comptabilité analytique}
\label{\detokenize{accounting/costaccounting:comptabilite-analytique}}\label{\detokenize{accounting/costaccounting::doc}}
Pour permettre de réaliser une analyse financière des différentes activités de votre structure, vous pouvez mettre en place une comptabilité analytique.

La comptabilité analytique proposée par le logiciel est une version simplifiée.
En effet, il n’est pas possible de ventiler une même ligne écriture sur plusieurs codes analytiques.


\subsection{Les codes analytiques}
\label{\detokenize{accounting/costaccounting:les-codes-analytiques}}
Depuis le menu \sphinxstyleemphasis{Financier/Comptabilité/Comptabilités analytiques}, vous accédez à la liste des codes.

Depuis cet écran vous pouvez créer, modifier ou supprimer un code analytique. Celui-ci est constitué
en plus d’un titre et d’un description, d’un status (Ouvert ou Clôturé).

Depuis cette liste, vous obtiendrez également le résultat comptable (les produits diminués des charges) de votre code analytique.
\begin{quote}

\noindent\sphinxincludegraphics{{costaccount_list}.png}
\end{quote}

Par défaut, un filtrage vous pemet de ne voir que les code analytique courant. Cliquez dans la coche pour désactiver ce filtre.

A noter que dans le menu \sphinxstyleemphasis{Administration/Modules (conf.)/Configuration comptable}, vous avez un paramètre vous permettant
de rendre obligatoire une affectation analytique à toutes charges ou produits.


\subsection{Imputation analytique d’une écriture}
\label{\detokenize{accounting/costaccounting:imputation-analytique-d-une-ecriture}}
Si vous avez des codes analytiques ouverts, vous pouvez imputer une ligne écriture sur l’un d’entre eux.
\begin{quote}

\noindent\sphinxincludegraphics{{costaccount_assign}.png}
\end{quote}

Pour cela, éditez votre écriture (validée ou non) à imputer à votre code analytique, et modifiez la ligne d’écriture avec l’affectation désirée.

Il est aussi possible de réaliser cette imputation par lot depuis la liste des écritures.
Pour cela, sélectionnez les écritures à affecter et cliquez sur \sphinxstyleemphasis{Analytique}: choisissez alors le nouveau code à utiliser
pour l’ensemble des lignes d’écritures de charges ou de produits.


\subsection{Impressions analytiques}
\label{\detokenize{accounting/costaccounting:impressions-analytiques}}
Depuis la liste des codes analytiques, vous pouvez réaliser un rapport type « compte de résultats ».

Ce rapport est les équivalents de ceux obligatoires pour une comptabilité sur un exercice mais adaptés au besoin d’une comptabilité analytique.


\section{Modèle}
\label{\detokenize{accounting/model:modele}}\label{\detokenize{accounting/model::doc}}

\subsection{Déclaration d’un modèle}
\label{\detokenize{accounting/model:declaration-d-un-modele}}
Un modèle d’écriture est un ensemble de lignes d’écritures mémorisées à l’avance que vous pouvez rejouer aussi souvent que vous le voulez.
Depuis le menu Financier/Comptabilité/Modèles d’écritures, vous accédez à la liste des modèles.
\begin{quote}

\noindent\sphinxincludegraphics{{model_list}.png}
\end{quote}

Un modèle est associé à un journal et contient une description ainsi qu’une liste de lignes comprenant un code comptable et un montant.
\begin{quote}

\noindent\sphinxincludegraphics{{model_item}.png}
\end{quote}

Nous vous conseillons de créer un modèle pour chacune de vos dèpenses (ou recette) règulières. Ainsi, vous gagnerez du temps sur la saisie de votre comptabilitè sans avoir à rechercher le bon code comptable.


\subsection{Utilisation d’un modèle}
\label{\detokenize{accounting/model:utilisation-d-un-modele}}
L’utilisation d’un modèle est très simple. Avec le bouton Modèle présent dans les liste d’écritures, une sélection de modèle vous est présentée.
\begin{quote}

\noindent\sphinxincludegraphics{{model_add}.png}
\end{quote}

Sélectionnez votre modèle et précisez un coefficient multiplicateur. Ce facteur est très pratique lorsque l’on a des factures récurrentes mais dont le montant peut fluctuer. Il est alors possible, è l’aide de ce rèel, de l’affiner.
Une fois validé, une écriture est générée suivant la description du modèle. Vous pouvez la corriger comme n’importe quelle écriture.


\section{Budget prévisionnel}
\label{\detokenize{accounting/budget::doc}}\label{\detokenize{accounting/budget:budget-previsionnel}}

\subsection{Budget par analytique}
\label{\detokenize{accounting/budget:budget-par-analytique}}
Depuis l’interface des comptabilités analytiques, vous pouvez ajouter un budget prévisionnel à chacun.
Cliquez simplement sur le bouton \sphinxstyleemphasis{Budget} après avoir sélectionner une comptabilité à compléter.

L’interface vous permet alors d’ajouter des comptes de charges ou de produits ainsi qu’un solde prévisionnel.
vous pouvez également importer les montants des charges et produits du résultat d’une comptabilités précédentes.

Ce budget prévisionnel apparait alors dans les rapports afin de le comparer avec la comptabilité réalisé.


\subsection{Budget par exercice}
\label{\detokenize{accounting/budget:budget-par-exercice}}
Depuis l’interface du plan comptable courant, vous pouvez ajouter un budget prévisionnel à l’exercice via le bouton \sphinxstyleemphasis{Budget}.

Comme pour le budget analytique, vous pouvez ajouter des comptes de charges ou de produits ainsi que d’importer le résultat de l’exercice précédent.
A noter qu’automatiquement, l’ensemble des budgets analytiques associés au même exercice sont automatiques consolidés dans ce budget d’exercice.

Le \sphinxstyleemphasis{resultat d’exercice} presente également la comptabilité courant en affichant également le budget prévisionnel à des fins de comparaison.


\section{Raports}
\label{\detokenize{accounting/reporting::doc}}\label{\detokenize{accounting/reporting:raports}}
Dans le catégorie \sphinxstyleemphasis{Financier/Comptabilité}, vous avez accès à un ensemble de rapports relatifs à votre comptabilité.
Vous pouvez consulter votre rapport ainsi qu’en réaliser une impression via une génération PDF de ce rapport.
De plus, pour tout les exercices non-cloturés, vous pouvez préciser une période de consultation.
Ceci est principalement utile pour vous aider à faire vos rapprochements bancaires ou si vous avez besoins d’éditer des situations financiéres trimestrielles.


\subsection{Compte de résultats}
\label{\detokenize{accounting/reporting:compte-de-resultats}}
Le compte de résultat est un document comptable synthétisant l’ensemble des charges et des produits d’une entreprise ou autre organisme ayant une activité marchande, au cours d’un son exercice comptable.
Ce document donne le résultat net, c’est-à-dire ce que l’entreprise a gagné (bénéfice) ou perdu (perte) au cours de la période.


\subsection{Bilan}
\label{\detokenize{accounting/reporting:bilan}}
Le bilan est une photographie du patrimoine de l’entreprise qui permet de réaliser une évaluation d’entreprise, et plus précisément de savoir après retraitement (par exemple d’une optique patrimoniale à celle sur option de juste valeur pour l’adoption des normes internationales) combien elle vaut et si elle est solvable.
Il existe donc trois finalités au bilan:
\begin{itemize}
\item {} 
Le bilan comptable interne, généralement détaillé, utilisé par les responsables de l’entreprise pour différentes analyses internes;

\item {} 
Le bilan comptable officiel, destiné aux contrôleurs de la comptabilité (auditeurs et commissaires aux comptes) et aux actionnaires (et plus généralement aux tiers);

\item {} 
Le bilan fiscal, qui sert à déterminer le bénéfice imposable;

\end{itemize}


\subsection{Grand livre}
\label{\detokenize{accounting/reporting:grand-livre}}
Le Grand livre est le recueil de l’ensemble des comptes utilisés d’une entreprise qui tient sa comptabilité en partie double.


\subsection{Balance}
\label{\detokenize{accounting/reporting:balance}}
La balance comptable est un état d’une période, établi à partir de la liste de tous les comptes du grand livre de l’entreprise (qu’ils soient de bilan ou de gestion) et regroupant tous les totaux (ou masses) en débit et crédit de ces comptes et par différence tous les soldes débiteurs et créditeurs.


\subsection{Listing des écritures}
\label{\detokenize{accounting/reporting:listing-des-ecritures}}
Depuis l’écran de la liste des écritures comptables, vous avez la possibilité d’exporter l’ensemble des écritures de l’exercice.
Vous pourrez visualiser, imprimer, exporter au format PDF ou CSV (permet l’import de vos écritures dans un tableur).


\subsection{Listing du plan comptable de l’exercice}
\label{\detokenize{accounting/reporting:listing-du-plan-comptable-de-l-exercice}}
Depuis l’écran du plan comptable de l’exercice, vous avez la possibilité d’exporter l’ensemble des écritures de code comptable utilisés et leur solde du moment.
Vous pourrez visualiser, imprimer, exporter au format PDF ou CSV (permet l’import de vos écritures dans un tableur).


\chapter{Diacamma règlement}
\label{\detokenize{payoff/index::doc}}\label{\detokenize{payoff/index:diacamma-reglement}}
Aide relative aux fonctionnalités de gestion des payements.


\section{Règlement}
\label{\detokenize{payoff/payoff::doc}}\label{\detokenize{payoff/payoff:reglement}}
Depuis un module tel que la facturation, il vous est possible de gérer leur règlement.

Depuis la fiche du document, cliquez sur «ajouter» un paiement.
\begin{quote}

\noindent\sphinxincludegraphics{{payoff1}.png}
\end{quote}

Vous pouvez alors saisir le mode de paiement de votre client ainsi que le compte bancaire à imputer de ce mouvement financier.

Dans la facture, vous pouvez consulter en plus de son montant total, la somme versée ainsi que le résidu de paiement à effectuer.

Chaque règlement génère automatiquement une écriture comptable dans le journal d’encaissement.

Il est aussi possible d’effectuer un seul règlement sur plusieurs document financier (comme les factures). Pour cela sélectionnez dans la liste des éléments « valides » celles que vous souhaitez et cliquez sur Réglement.
\begin{quote}

\noindent\sphinxincludegraphics{{multi-payoff}.png}
\end{quote}

Suivant le type de document sur lequel ce paiement est associé, vous pouvez avoir plusieurs modes de répartition:
\begin{itemize}
\item {} 
Par date
Ce paiement est d’abort ventilé sur le document financier le plus ancien, puis le suivant, etc.

\item {} 
Par prorata
Ce paiement multiple sera automatique ventilé sur document financier au prorata de leur montant.

\end{itemize}

Dans tout les cas, une seule écriture comptable d’encaissement sera alors réalisée.


\section{Dépôt de chèques}
\label{\detokenize{payoff/deposit:depot-de-cheques}}\label{\detokenize{payoff/deposit::doc}}
Depuis le menu \sphinxstyleemphasis{Finance/Dépôt de chèques}, vous pouvez ajouter et consulter des bordereaux de chèques.
\begin{quote}

\noindent\sphinxincludegraphics{{depositlist}.png}
\end{quote}

Dans une fiche de bordereau, vous pouvez sélectionner les règlements effectués par chèques dans vos différentes factures.
Cela vous constitue une liasse de chèques que vous pourrez déposer à votre agence bancaire.
Une fois réalisée, clôturez la sélection définie.
\begin{quote}

\noindent\sphinxincludegraphics{{deposititem}.png}
\end{quote}

Vous pouvez alors imprimer le bordereau de remise de chèques que vous pouvez joindre à votre liasse lors du dépôt de celle-ci.
De plus, une fois que votre bordereau apparaît sur votre relevé de compte, vous pouvez valider l’ensemble de vos écritures comptables depuis la fiche elle-même.


\section{Configuration}
\label{\detokenize{payoff/config::doc}}\label{\detokenize{payoff/config:configuration}}
Le menu \sphinxstyleemphasis{Administration/Configuration du règlement} vous permet quelques configurations pour votre structure.


\subsection{Compte bancaire}
\label{\detokenize{payoff/config:compte-bancaire}}
Dans cet écran, vous avez la possibilité d’enregistrer vos différents comptes bancaires que vous possédez.
Pour chacun, vous pouvez saisir l’intégralité des informations figurant sur un RIB.
Cela vous permettra d’éditer un résumé complet de vos dépôts de chèques.


\subsection{Moyen de paiement}
\label{\detokenize{payoff/config:moyen-de-paiement}}
Vous pouvez ici préciser les moyens de paiement que vous supportez.
Actuellement, 3 moyens de paiement sont pris en compte par \sphinxstyleemphasis{Diacamma}
\begin{itemize}
\item {} 
Le virement bancaire

\item {} 
Le chèque

\item {} 
Le paiement PayPal

\end{itemize}

Pour chacun d’entre eux, vous devez préciser les paramètres relatifs.

C’est moyen de paiement peuvent être utilisé pour vos débiteurs afin de régler par un de ses moyens ce qu’ils vous doivent.

Dans le cas de PayPal, si votre \sphinxstyleemphasis{Diacamma} est accessible par internet, le logiciel peux être notifié directement du paiement et ajouter un réglement correspondant directement dans votre logiciel.
Il est conseillé, dans ce cas, de cocher le champ \sphinxstyleemphasis{avec contrôle}: le lien de paiement présenter dans un courriel redirigera alors en premier sur votre \sphinxstyleemphasis{Diacamma} afin de vérifier que cet élément financier est toujours d’actualité.


\subsection{Paramètres}
\label{\detokenize{payoff/config:parametres}}
2 Paramètres actuellements:
\begin{itemize}
\item {} 
compte de caisse: indique le code comptable à imputer pour les règlements en espèce.

\item {} 
compte de frais bancaire: prècise un code comptable pour imputer directement, suite à un règlement, des frais bancaires inhérent à ce règlement.

\end{itemize}

Un ligne d’écriture est alors ajouté directement à l’écriture comptable correspondant.
Si ce code est vide, aucun frais bancaire ne vous sera demandé.


\chapter{Lucterios contacts}
\label{\detokenize{contacts/index::doc}}\label{\detokenize{contacts/index:lucterios-contacts}}
Aide relative aux fonctionnalités de gestion de contacts moraux ou physiques.


\section{Les contact physiques}
\label{\detokenize{contacts/individual:les-contact-physiques}}\label{\detokenize{contacts/individual::doc}}
Un contact physique est une personne, homme ou femme, à mémoriser.


\subsection{Liste de vos contacts physiques}
\label{\detokenize{contacts/individual:liste-de-vos-contacts-physiques}}
Le menu \sphinxstyleemphasis{Bureautique/Adresses et Contacts/Personnes Physiques} vous permet de consulter la liste des personnes que vous avez déjà enregistrées. Étant donné que la liste peut devenir importante, il est possible de filtrer les personnes par leur nom.

Depuis cet écran, vous avez aussi la possibilité d’imprimer la liste des personnes.

\noindent\sphinxincludegraphics{{ListIndividual}.png}


\subsection{Edition d’un contact physique}
\label{\detokenize{contacts/individual:edition-d-un-contact-physique}}
Depuis la liste précédente, vous avez la possibilité d’ajouter une nouvelle personne. Vous pouvez ré-éditer cette fiche depuis sa visualisation.

\noindent\sphinxincludegraphics{{EditIndividual}.png}


\subsection{Visualisation d’un contact physique}
\label{\detokenize{contacts/individual:visualisation-d-un-contact-physique}}
Depuis la liste des personnes physiques, vous avez la possibilité de visualiser une personne.

Cela vous permettra de consulter la fiche d’une personne précédemment enregistrée dans votre base. Vous pouvez modifier cette fiche ou l’imprimer. Vous pouvez également lui donner un alias de connexion à l’application associé à un droit d’accès (voir Les utilisateurs). Si cette personne n’est pas référencée dans d’autres enregistrements de l’application, vous avez la possibilité de la supprimer.

\noindent\sphinxincludegraphics{{ShowIndividual}.png}


\subsection{Recherche d’un contact physique}
\label{\detokenize{contacts/individual:recherche-d-un-contact-physique}}
Le menu Bureautique/Adresses et Contacts/Recherche de personne physique de personne physique vous permet de définir un critère de recherche sur une personne physique.

Une fois validé, l’outil va rechercher dans la base toutes les personnes correspondantes à ces critères. Vous pourrez alors imprimer cette liste ou en visualiser/modifier une fiche.

\noindent\sphinxincludegraphics{{FindIndividual}.png}


\section{Les contacts moraux}
\label{\detokenize{contacts/legal_entity::doc}}\label{\detokenize{contacts/legal_entity:les-contacts-moraux}}
Un contact moral est une structure ou d’une organisation de personne (entreprise, association, administration, …), à mémoriser.


\subsection{Liste de vos contacts moraux}
\label{\detokenize{contacts/legal_entity:liste-de-vos-contacts-moraux}}
Le menu \sphinxstyleemphasis{Bureautique/Adresses et Contacts/Personnes morales} vous permet de consulter la liste des structures que vous avez déjà enregistrées. Chaque contact moral est associé à une catégorie. Dans cette liste, vous consultez vos structures filtrées par ces catégories.

Depuis cet écran, vous avez aussi la possibilité d’imprimer la liste des structures.

\noindent\sphinxincludegraphics{{ListLegalEntity}.png}


\subsection{Edition d’un contact moral}
\label{\detokenize{contacts/legal_entity:edition-d-un-contact-moral}}
Depuis la liste précédente, vous avez la possibilité de créer une nouvelle structure. Vous pouvez ré-éditer cette fiche depuis sa visualisation.

\noindent\sphinxincludegraphics{{EditLegalEntity}.png}


\subsection{Visualisation d’un contact moral}
\label{\detokenize{contacts/legal_entity:visualisation-d-un-contact-moral}}
Depuis la liste des personnes morales, vous avez la possibilité de visualiser une structure.

Cela vous permettra de consulter la fiche d’une structure précédemment entrée dans votre base. Vous pouvez modifier cette fiche ou l’imprimer. Si cette personne n’est pas référencée dans d’autre enregistrement de l’application, vous avez la possibilité de la supprimer.

\noindent\sphinxincludegraphics{{ShowLegalEntity}.png}


\subsection{Responsables d’un contact moral}
\label{\detokenize{contacts/legal_entity:responsables-d-un-contact-moral}}
Vous avez la possibilité d’associer une personne physique à votre structure.

Choisissez le nouveau responsable: si la personne n’existe pas dans votre base, vous aurez la possibilité de la créer. Vous pourrez également ajouter une fonction à un responsable défini.

\noindent\sphinxincludegraphics{{ResponsabilityLegalEntity}.png}


\subsection{Recherche d’un contact moral}
\label{\detokenize{contacts/legal_entity:recherche-d-un-contact-moral}}
Le menu \sphinxstyleemphasis{Bureautique/Adresses et Contacts/Recherche de personne morale} vous permet de définir un critère de recherche sur une structure morale.

\noindent\sphinxincludegraphics{{FindLegalEntity}.png}


\section{Configuration et paramétrage}
\label{\detokenize{contacts/configuration:configuration-et-parametrage}}\label{\detokenize{contacts/configuration::doc}}
Dans le menu \sphinxstyleemphasis{Administration/Modules (conf.)} vous avez à votre disposition des outils pour configurer la gestion des contacts.


\subsection{Configuration des contacts}
\label{\detokenize{contacts/configuration:configuration-des-contacts}}
Dans cet écran, vous avez la possibilité de créer ou de modifier une définition de fonction, ou responsabilité, pour associer une personne physique à une structure morale. Vous pouvez créer ou modifier une catégorie de structure morale pour vous aider dans la classification de vos contacts moraux.

Il se peut que vous ayez besoin de préciser des informations supplémentaires pour vos différents contacts. Vous avez ici la possibilité d’ajouter des champs personnels pour chaque type de contacts. Pour ajouter un champ, vous devez simplement donner son titre ainsi que définir son type et éventuellement les compléments nécessaires.
5 types are possibles:
\begin{itemize}
\item {} 
chaîne de texte

\item {} 
nombre entier

\item {} 
nombre à virgule (réel)

\item {} 
valeur Oui/Non (booléen)

\item {} 
choix dans une liste (énumération)

\end{itemize}

Dans le cas de l’énumération, vous devez définir la liste des valeurs possibles (mots) séparées par un point-virgule.


\subsection{Codes postaux/villes}
\label{\detokenize{contacts/configuration:codes-postaux-villes}}
Cela peux vous aider dans votre saisi de contact, l’outil va automatiquement rechercher la ville (ou les villes) associée(s) avec le code postal que vous entrerez.
Dans cet écran, vous pouvez ajouter des codes postaux manquants.
Par défaut, les codes postaux français et suisses sont insérés.


\chapter{Lucterios courier}
\label{\detokenize{mailing/index::doc}}\label{\detokenize{mailing/index:lucterios-courier}}
Aide relative aux fonctionnalités de courier et publipostage.


\section{Configuration du couriel}
\label{\detokenize{mailing/configuration::doc}}\label{\detokenize{mailing/configuration:configuration-du-couriel}}
Vous pouvez configurer ici des réglages pour l’envoi de couriel.

Le serveur SMTP permettra au logiciel d’envoyer directement des messages à vos contacts comme par exemple l’envoi d’un nouveau mot de passe de connexion. N’oubliez pas alors de préciser un petit message d’explication.

Vous pouvez aussi préciser comment vous voulez utiliser votre navigateur de couriel pour la fonctionnalité “écrire à tous”: “Pour”, “Copie à” ou “Copie caché à”.


\section{Publipostage}
\label{\detokenize{mailing/mailing:publipostage}}\label{\detokenize{mailing/mailing::doc}}
Depuis le menu \sphinxstyleemphasis{Bureatique/Publipostage/Message} vous avez la possibilité de créer un courier de publipostage.

Une fois votre message rédigé, vous pouvez lui associé des requetes de destinataires.
C’est requetes de recherches, similaire à celle des outils de recherche de contacts, ne seront évaluées qu’au moment de la génération du courier.
Ainsi, même un contact dernièrement ajouté ou modifié pourra être impacté par ce message.

\noindent\sphinxincludegraphics{{mailing}.png}
\begin{description}
\item[{Une fois le message validé vous pouvez:}] \leavevmode\begin{itemize}
\item {} 
Soit généré une sortie PDF de l’ensemble des lettres à envoyer personnalisé avec l’entête de chaque contact

\item {} 
Soit envoyé par courriel si votre configuration est valide. Bien sur, dans ce cas, seul les contacts possédant une adresse seront impacté par cet envoie.

\end{itemize}

\end{description}

Il est également possible d’ajouter à votre message un ou plusieurs documents, sauvés dans le \sphinxstyleemphasis{gestionnaire de documentation}.
Ces documents seront transmis en pièces-jointes dans l’envoie par courriel.

L’option \sphinxstyleemphasis{document(s) ajouté(s) via liens dans le message} permet d’ajouter un ensemble de liens partagés vers vos documents (et non plus des pièces jointes).
Cela permet de gérer des documents de taille importante ou qui risqueraient d’être supprimer par certain gestionnaire de courriel.


\chapter{Lucterios documents}
\label{\detokenize{documents/index::doc}}\label{\detokenize{documents/index:lucterios-documents}}
Aide relative aux fonctionnalités de gestion documentaire.


\section{Fichiers partagés}
\label{\detokenize{documents/shared_document:fichiers-partages}}\label{\detokenize{documents/shared_document::doc}}

\subsection{Liste des documents}
\label{\detokenize{documents/shared_document:liste-des-documents}}
Le menu \sphinxstyleemphasis{Bureautique/Gestion documentaire/Documents} vous permet de consulter la liste des fichiers que vous avez déjà enregistrés. Pour vous aider à retrouver vos documents, la liste est classifiée par un ensemble de dossiers et de sous-dossiers et une description vous donne un petit résumé.

Vous avez aussi la possibilité d’ajouter un sous-dossier ou de modifier les propriétés du dossier courant.

\noindent\sphinxincludegraphics{{listdoc}.png}

Suivant vos permissions, vous pouvez extraire votre fichier pour le consulter, le modifier et éventuellement ré-injecter vos corrections.

De plus, l’outil mémorisera l’utilisateur et la date de création du document ainsi que les informations relatives à la dernière modification.

\noindent\sphinxincludegraphics{{showdoc}.png}

Depuis la fiche du document, il vous est possible d’activer un lien de téléchargement.
Ce lien web peut être transmis à une personne tiers, n’ayant aucun droit d’accès à votre logiciel, afin de télécharger le document.
\sphinxstylestrong{Attention:} Votre instance doit être accessible sur internet pour que le lien puisse fonctionner depuis n’importe où.


\subsection{Recherche de documents}
\label{\detokenize{documents/shared_document:recherche-de-documents}}
Le menu \sphinxstyleemphasis{Bureautique/Gestion documentaire/Recherche de document} vous permet de définir un critère de recherche sur un document.

Une fois validé, l’outil va rechercher dans la base toutes les fichiers correspondants à ces critères.


\section{Configuration}
\label{\detokenize{documents/configuration::doc}}\label{\detokenize{documents/configuration:configuration}}
Dans le menu \sphinxstyleemphasis{Administration/Module (conf)/Dossiers} vous avez à votre disposition un ensemble d’outils pour configurer la gestion documentaire.


\subsection{Dossiers}
\label{\detokenize{documents/configuration:dossiers}}
Dans cet écran vous avez la possibilité de créer ou de modifier des dossiers de classement documentaire.

\noindent\sphinxincludegraphics{{configuration}.png}

En associant judicieusement un dossier comme sous-dossier d’un parent, vous pouvez vous définir une arborescence de classement.

Vous associez à chaque dossier un ensemble de groupe de droits pour la visualisation et la modification des fichiers. Seuls les utilisateurs appartenant aux groupes de visualisation pourront consulter les documents de cette catégorie. Seuls les utilisateurs appartenant aux groupes de modification pourront corriger les documents de cette catégorie.


\chapter{Coeur Lucterios}
\label{\detokenize{CORE/index::doc}}\label{\detokenize{CORE/index:coeur-lucterios}}
Aide relative aux fonctionnalités générales de cet outil de gestion.


\section{Mot de passe}
\label{\detokenize{CORE/password::doc}}\label{\detokenize{CORE/password:mot-de-passe}}
Le menu \sphinxtitleref{Général/Mot de passe} vous permet de changer le mot de passe d’accès de l’utilisateur courant.

\noindent\sphinxincludegraphics{{password}.png}

Pour plus de sécurité, nous vous conseillons d’utiliser un mot de passe comprenant des lettres et des chiffres et ne constituant pas un mot compréhensible.


\section{Les groupes}
\label{\detokenize{CORE/groups:les-groupes}}\label{\detokenize{CORE/groups::doc}}
Le menu \sphinxtitleref{Administration/Gestion des Droits/Les groupes} vous permet de créer, modifier ou supprimer un groupe de droits.

\noindent\sphinxincludegraphics{{group}.png}

Un groupe de droits réunit un ensemble d’autorisations aux actions de l’application.

\noindent\sphinxincludegraphics{{group_modify}.png}


\section{Les utilisateurs}
\label{\detokenize{CORE/users::doc}}\label{\detokenize{CORE/users:les-utilisateurs}}
Le menu \sphinxtitleref{Administration/Gestion des Droits/Les utilisateurs} vous
permet de créer, modifier ou désactiver un utilisateur de l’application. Un
utilisateur définit un droit de connexion au logiciel.

\noindent\sphinxincludegraphics{{users}.png}

Depuis cette liste, vous pouvez créer ou modifier l’utilisateur: son
alias, son nom et son mot de passe. A cela, vous lui ajouter des groupes et
des permissions suplémentaires éventuelles afin de définir son niveau
d’accès au logiciel. Vous pouvez aussi désactiver un utilisateur pour lui
interdire l’accès à l’application.

\noindent\sphinxincludegraphics{{user_info}.png}

\noindent\sphinxincludegraphics{{user_permissions}.png}


\section{L’architecture du logiciel}
\label{\detokenize{CORE/architecture::doc}}\label{\detokenize{CORE/architecture:l-architecture-du-logiciel}}
Depuis le commencement de ce logiciel, les développeurs ont voulu que cette application puisse avoir une architecture ouverte permettant des évolutions les plus larges.



\renewcommand{\indexname}{Index}
\printindex
\end{document}